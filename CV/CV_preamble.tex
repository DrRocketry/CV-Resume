%Eventually, I will organize this into related categories...

\usepackage[english]{babel}					%All generated text BELOW THIS COMMAND will be in the specified language. i.e., you must put lipsum and blindtext below this. 
\usepackage[margin=1in]{geometry}		%for people who fuss about things they don't understand. Set your margins here. Remember to leave room for the header, if you have one.
\usepackage{amsmath,amsfonts,amsthm,amssymb}			%American Mathematical Society packages 
\usepackage[iso]{datetime}					%changes \today to ISO format, includes other customization options
\usepackage{graphicx}						%used for inserting pictures
%\usepackage{caption}
%\usepackage{subcaption}
%\usepackage{calc}						%used for performing arithmetic operations in arguments. for example: \textwidth/2
\usepackage{color}						%used for coloring text
%\usepackage{blindtext}						%used for testing
%\usepackage{lipsum}
%%\usepackage{siunitx}	%not on MCECS computers			%proper formatting for numbers and units
\usepackage{hyperref}						%used for web and intra-document links
%\usepackage{fancyvrb}						%Makes fancy Verbatim (must use capital V) environments
%\usepackage{esint}						%Has closed double integrals: \oiint
%%\usepackage[round, numbers, sort]{natbib}				%used for citations with bibtex
%\usepackage{float}						%does fancy things with floats, like put boxes around figures
%\usepackage[none]{hyphenat}					%removes hyphenation (for copying into a word processor)
%\usepackage{multicol}	% allows multi-column sections (useful for lists with short items)
%\usepackage[citestyle=verbose]{biblatex}	%citations
%\usepackage[backend=bibtex]{biblatex}
%	\bibliography{CV.bib}

%\bibliography{CV.bib}
%\usepackage{bibentry}
